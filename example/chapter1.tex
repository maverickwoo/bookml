\chapter{The first chapter}

Two images below, demonstrating \texttt{bmlimage} and \texttt{bmlImageEnvironment} (see preamble) for when LaTeXML is not able to produce the correct image. Note also \texttt{\textbackslash{}bmlDescription} used to add an alternative text description to the images.
\begin{figure}
\[ \begin{bmlimage}
  \xymatrix{
    U \ar@/_/[ddr]_y \ar@/^/[drr]^x \ar@{.>}[dr]|-{(x,y)} \\
    & X \times_Z Y \ar[d]^q \ar[r]_p & X \ar[d]_f \\
    & Y \ar[r]^g & Z}
  \end{bmlimage}\bmlDescription{Diagram representing the pull-back of two functions.} \]
  \caption{Example of \texttt{xymatrix} from the \texttt{xypic} documentation.}
\end{figure}

\begin{figure}
  \begin{center}
    \begin{tikzcd}
      A \arrow[rd] \arrow[r, "\phi"] & B \\
                                     & C
    \end{tikzcd}
    \bmlDescription{A, B, C drawn in a triangle with C under B, an arrow labelled phi from A to B and an arrow from A to C}
  \end{center}
  \caption{Example of \texttt{tikzcd} diagram.}
\end{figure}

For more sophisticated functionality, see the \href{https://vlmantova.github.io/bookml/}{BookML manual} or read its \href{https://github.com/vlmantova/bookml/blob/docs/docs.tex}{\TeX{} source}.

By default, BookML asks LaTeXML to split each section into a separate page. Use different values of \texttt{SPLITAT} to get different results (remember to delete the generated files before running \texttt{make} a second time!).

\section{First section}

Each section will normally appear on a different page.

\section{Second section}

Each section will normally appear on a different page.
