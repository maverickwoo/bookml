\documentclass{book}

% required by BookML: use T1 font enconding
\usepackage[T1]{fontenc}

% required for accessibility: set the language for the output (in this case, en-GB)
\usepackage[british]{babel}

% required for correct formatting and tagging of BookML output: title and author
\title{A minimal BookML example}
\author{Vincenzo Mantova}

% optional but recommended: additional BookML and LaTeXML functionality
\usepackage{bookml/bookml}

% optional but recommended: enable clickable links in PDFs
% the option pdfusetitle adds \author and \title to the PDF metadata
\usepackage[pdfusetitle]{hyperref}

% recommended: use bmlImageEnvironment for image heavy content, such as TikZ, where LaTeXML is slow and often produces garbled output
% this command declares tikzpicture and tikzcd environments to be compiled via LaTeX instead of LaTeXML
\bmlImageEnvironment{tikzpicture,tikzcd}

% this hides the tikz, xy packages and some tikz-related commands from LaTeXML
\iflatexml
\else
\usepackage{tikz}
\usetikzlibrary{cd}
\usepackage[all]{xy}
\fi

\begin{document}

% example of footer, here for a copyright notice
% note that the footer does not appear in the PDF -- it must be recreated with e.g. fancyhdr if needed
\begin{lxFooter}
  Example of footer: copyright \copyright{} 2022 Vincenzo Mantova.
\end{lxFooter}

To compile this document, simply run \texttt{make} in this folder (or possibly \texttt{gmake} if you are on Windows). BookML will generate a file \texttt{\jobname.zip} containing the HTML output generated by LaTeXML. You can also view the output in \texttt{\jobname/index.html}.

By default, each section will create a separate page. Try running \texttt{make SPLITAT=} to have all sections in a single file, or \texttt{make SPLITAT=chapter} to have a separate page per chapter. If you do so, run \texttt{make clean} first, or manually delete the folder \texttt{\jobname} and \texttt{\jobname.zip}.

You may also modify \texttt{Makefile} to permanently change how pages are split, or to indicate a different splitting strategy for different zip packages.

\textbf{Requirements:} you must have \texttt{latexmk} in your MikTeX/TeXLive distribution, GNU Make, \texttt{zip} on macOS and Linux or 7-zip on Windows, and of course LaTeXML.

\textbf{Multiple files:} note that you will find a \texttt{secondfile.zip} file as well. Each \texttt{.tex} file in this folder containing \texttt{\textbackslash{}documentclass} gets its own zip package. Try creating a new \LaTeX{} document \texttt{thirdfile.tex} and run \texttt{make} again.

\chapter{The first chapter}

Two images below, demonstrating \texttt{bmlimage} and \texttt{bmlImageEnvironment} (see preamble) for when LaTeXML is not able to produce the correct image. Note also \texttt{\textbackslash{}bmlDescription} used to add an alternative text description to the images.
\begin{figure}
\[ \begin{bmlimage}
  \xymatrix{
    U \ar@/_/[ddr]_y \ar@/^/[drr]^x \ar@{.>}[dr]|-{(x,y)} \\
    & X \times_Z Y \ar[d]^q \ar[r]_p & X \ar[d]_f \\
    & Y \ar[r]^g & Z}
  \end{bmlimage}\bmlDescription{Diagram representing the pull-back of two functions.} \]
  \caption{Example of \texttt{xymatrix} from the \texttt{xypic} documentation.}
\end{figure}

\begin{figure}
  \begin{center}
    \begin{tikzcd}
      A \arrow[rd] \arrow[r, "\phi"] & B \\
                                     & C
    \end{tikzcd}
    \bmlDescription{A, B, C drawn in a triangle with C under B, an arrow labelled phi from A to B and an arrow from A to C}
  \end{center}
  \caption{Example of \texttt{tikzcd} diagram.}
\end{figure}

For more sophisticated functionality, see the \href{https://vlmantova.github.io/bookml/}{BookML manual} or read its \href{https://github.com/vlmantova/bookml/blob/docs/docs.tex}{\TeX{} source}.

By default, BookML asks LaTeXML to split each section into a separate page. Use different values of \texttt{SPLITAT} to get different results (remember to delete the generated files before running \texttt{make} a second time!).

\section{First section}

Each section will normally appear on a different page.

\section{Second section}

Each section will normally appear on a different page.


\end{document}
